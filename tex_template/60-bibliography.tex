\section*{СПИСОК ИСПОЛЬЗОВАННЫХ ИСТОЧНИКОВ}
\addcontentsline{toc}{section}{СПИСОК ИСПОЛЬЗОВАННЫХ ИСТОЧНИКОВ}

\begingroup
\renewcommand{\section}[2]{}
\begin{thebibliography}{}

	\bibitem{novainfo}
	Научный журнал NovaInfo [Электронный ресурс]. - Режим доступа: https://novainfo.ru, свободный. –
	(дата обращения: 08.11.2021)
	
	\bibitem{archives}
	Черепанов Н. В., Буслаев С. П. Проблемы создания электронного архива конструкторской документации на основе бумажного архива //Инновации и инвестиции. – 2020. – №. 10.
	
	\bibitem{archives2}
	Етрукова Д. А. ТЕХНОЛОГИИ ПЕРЕВОДА БУМАЖНЫХ ДОКУМЕНТОВ В ЭЛЕКТРОННЫЙ ВИД //Инженерные кадры-будущее инновационной экономики России. – 2019. – №. 6. – С. 44-47.
	
	\bibitem{translate}
	Дроздова К. А. Машинный перевод: история, классификация, методы //Вестник Омского государственного педагогического университета. Гуманитарные исследования. – 2015. – №. 3 (7).
	
	\bibitem{yandexTranslateInfo}
	О сервисе - Яндекс.Переводчик.Справка  [Электронный ресурс]. - Режим доступа: https://yandex.ru/support/translate/, свободный. – (дата обращения: 08.12.2021)
	
	\bibitem{yandexTranslateLanguages}
	Яндекс.Переводчик.Справка - Список поддерживаемых языков [Электронный ресурс]. - Режим доступа: https://yandex.ru/support/translate/supported-langs.html, свободный. – (дата обращения: 08.12.2021)
	
	\bibitem{yandexTranslateDictionary}
	Яндекс.Переводчик.Справка - Словарь [Электронный ресурс]. - Режим доступа: https://yandex.ru/support/translate/dictionary.html, свободный. – (дата обращения: 08.12.2021)
	
	\bibitem{abbyy}
	Как распознать текст в PDF = ABBYY Finereader PDF 15 [Электронный ресурс]. - Режим доступа: https://pdf.abbyy.com/ru/learning-center/how-to-recognize-text-in-pdf/, свободный. – (дата обращения: 08.12.2021)
	
	\bibitem{abbyyLanguages}
	Технические спецификации ABBYY FineReader PDF [Электронный ресурс]. - Режим доступа: https://pdf.abbyy.com/ru/specifications/, свободный. – (дата обращения: 08.12.2021)
	
	\bibitem{googleTranslate}
	App Store:Google Переводчик [Электронный ресурс]. - Режим доступа: https://apps.apple.com/ru/app/google-переводчик/id414706506, свободный. – (дата обращения: 08.12.2021)
	
	\bibitem{OCR}
	Mithe R., Indalkar S., Divekar N. Optical character recognition //International journal of recent technology and engineering (IJRTE). – 2013. – Т. 2. – №. 1. – С. 72-75.
	
	\bibitem{OCR2}
	Гришанов К. М., Белов Ю. С. Методы выделения признаков для распознавания символов //Электронный журнал: наука, техника и образование. – 2016. – №. 1. – С. 110-119.
	
	\bibitem{OCR3}
	Галуза И. В., Кузнецова А. В. ОПТИЧЕСКОЕ РАСПОЗНАВАНИЕ СИМВОЛОВ //Красноярск, Сибирский федеральный университет, 15-25 апреля 2016 г. – С. 44.
	
	\bibitem{OCRHistory}
	Memon J. et al. Handwritten optical character recognition (OCR): A comprehensive systematic literature review (SLR) //IEEE Access. – 2020. – Т. 8. – С. 142642-142668.
	
	\bibitem{OCR0}
	Islam N., Islam Z., Noor N. A survey on optical character recognition system //arXiv preprint arXiv:1710.05703. – 2017.
\end{thebibliography}
\endgroup

\pagebreak