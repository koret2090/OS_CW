\section*{ВВЕДЕНИЕ}
\addcontentsline{toc}{section}{ВВЕДЕНИЕ}
	%Ядро ОС Linux способно к модификации за счёт расширения функциональных возможностей. Это достигается несколькими способами, но самым оптимальным является подключение загружаемых модулей ядра. Этот способ позволяет вести разработку нужного функционала для ядра независимо от самого ядра.\par
	%https://itproffi.ru/zagruzhaemye-moduli-yadra-linux/
	%https://losst.ru/moduli-yadra-linux
	%В качестве примера может быть создан загружаемый модуль ядра для геймпада. 
	Одной из часто встречающихся задач является изменение функциональности внешнего устройства. Не всегда удобно пользоваться устройствами только в качестве их классического предназначения.\par
	Например, хотелось бы иметь возможность устройство наподобие геймпада использовать в качестве компьютерной мыши, для этого нужно разработать загружаемый модуль ядра, а именно: USB-драйвер, который способен решить поставленную задачу.\par
	
	
	%Цель данной курсовой работы - разработать загружаемый модуль ядра Linux, который позволяет использовать функционал мыши при помощи геймпада. Реализовать аналоги левого и правого кликов мыши, прокрутки колеса и перемещение курсора.\par
	Таким образом, данная работа посвящена реализации задачи изменения функциональности геймпада с целью использования его, как компьютерной мыши.

	
	%Геймпад — тип игрового манипулятора. Представляет собой пульт, который удерживается двумя руками, для контроля его органов управления используются большие пальцы рук (в современных геймпадах также часто используются указательные и средние пальцы).\par
	
	
\pagebreak