\section*{ВВЕДЕНИЕ}
\addcontentsline{toc}{section}{ВВЕДЕНИЕ}
	Системное ядро Linux способно к модификации за счёт расширения функциональных возможностей. Это достигается несколькими способами, но самым оптимальным является подключение загружаемых модулей ядра. Этот способ позволяет вести разработку нужного функционала для ядра независимо от самого ядра.\par
	%https://itproffi.ru/zagruzhaemye-moduli-yadra-linux/
	Модуль ядра - это программа, которая может быть загружена в ядро операционной системы, или выгружена из него по запросу без перекомпиляции ядра или перезагрузки системы. Модули предназначены для расширения функциональности ядра. Ядро Linux позволяет драйверам оборудования, файловых систем, и некоторым другим компонентам быть скомпилированными отдельно --- как модули, а не как часть самого ядра. Таким образом, вы можете обновлять драйвера для различных устройств не пересобирая ядро, а также динамически расширять его функциональность.\par
	%https://losst.ru/moduli-yadra-linux
	В качестве примера может быть создан загружаемый модуль ядра для геймпада. 
	Геймпад — тип игрового манипулятора. Представляет собой пульт, который удерживается двумя руками, для контроля его органов управления используются большие пальцы рук (в современных геймпадах также часто используются указательные и средние пальцы).\par
	Цель данной курсовой работы - разработать загружаемый модуль ядра Linux, который позволяет использовать функционал мыши при помощи геймпада. Реализовать аналоги левого и правого кликов мыши, прокрутки колеса и перемещение курсора.\par
	
	
	
\pagebreak