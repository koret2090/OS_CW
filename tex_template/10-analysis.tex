\section{Аналитическая часть}
\subsection{Постановка задачи}
Цель данной работы является разработка и реализация загружаемого
модуля ядра операционной системы Linux, который позволяет использовать функционал мыши при помощи геймпада. Реализовать аналоги левого и правого кликов мыши, прокрутки колеса и перемещение курсора. Для достижения
поставленной цели следует решить следующие задачи:
\begin{itemize}
	\item выполнить анализ системного драйвера геймпада;
	\item разработать загружаемый модуль, позволяющий использовать функционал компьютерной мыши
	при помощи геймпада;
	\item реализовать программное обеспечение.
\end{itemize}
Программное обеспечение должно позволять использовать клавиши геймпада, как аналоги левой и правой клавиш мыши, перемещать курсор при помощи левого стика, пролистывать страницы при помощи правого.\par

\subsection{Загружаемый модуль ядра}
Несмотря на то, что ядро Linux является монолитным, оно позволяет
выполнять динамическую вставку и удаление кода ядра в процессе работы.\par
Загружаемый объект ядра называется модулем. Модуль по своей сути примерно то же, что и обычная программа. Модуль
так же имеет точку входа и выхода и находится в своем бинарном файле. Но
модули имеют непосредственный доступ к структурам и функциям ядра. Для
программ в пространстве пользователя этот доступ ограничен библиотечными
интерфейсами компилятора.\par
%Соловьев А. Разработка модулей ядра ОС Linux Kernel newbie's manual.
Использование загружаемых модулей значительно упрощает изменение
функциональности ядра и не требует ни полной перекомпиляции, ни перезагрузок.\par
Также преимущество загружаемых модулей заключается в возможности сократить расход памяти для ядра, загружая только
необходимые модули.

\subsection{USB ядро и USB драйвер}
Universal Serial Bus (USB, Универсальная Последовательная Шина) является соединением
между компьютером и несколькими периферийными устройствами. Первоначально она была
создана для замены широкого круга медленных и различных шин, параллельной,
последовательной и клавиатурного соединений, на один тип шины, чтобы к ней могли
подключаться все устройства\par
%Corbet J., Rubini A., Kroah-Hartman G. Драйверы устройств Linux, Третья редакиця.

USB Core — это подсистема ядра Linux, созданная для поддержки USB-устройств и контроллеров шины USB. Ядро USB предоставляет интерфейс для драйверов USB, используемый для доступа и
управления USB оборудованием, без необходимости беспокоится о различных типах
аппаратных контроллеров USB, которые присутствуют в системе.\par

Ядро Linux поддерживает два основных типа драйверов USB: драйверы на хост-системе и
драйверы на устройстве. USB драйверы для хост-системы управляют USB-устройствами,
которые к ней подключены, с точки зрения хоста (обычно хостом USB является персональный
компьютер.) USB-драйверы в устройстве контролируют, как одно устройство видит хост-компьютер в качестве устройства USB.\par
Драйверы основного ядра обращаются к прикладным
интерфейсам USB ядра. В тоже время принято выделять два основных
публичных прикладных интерфейса: один --- реализует взаимодействие с
драйверами общего назначения (символьное устройство), другой ---
взаимодействие с драйверами, являющимися частью ядра (драйвер хаба).
Второй тип драйверов участвует в управлении USB шиной.\par
На Рисунке \ref{USB-device} 
представлена, как USB-устройства состоят из конфигураций, интерфейсов и оконечных точек и как USB
драйверы связаны с интерфейсами USB, а не всего устройства USB.

\begin{figure}[h!]
	\centering
	\includegraphics[scale=0.9]{img/Usb-device.pdf}
	\caption{Схема взаимодействия устройства и драйвера}
	\label{USB-device}
\end{figure}\par

\subsection{Оконечная точка}
Самый основной формой USB взаимодействия является то, что называется endpoint
(оконечная точка). Оконечная точка USB может переносить данные только в одном
направлении, либо со стороны хост-компьютера на устройство (называемая оконечной точкой
OUT) или от устройства на хост-компьютер (называемая оконечной точкой IN). Оконечные
точки можно рассматривать как однонаправленные трубы.\par
Драйвер геймпада имеет только 1 конечную точку типа прерывания. Для
конечных точек данного типа характерна передача небольшого объема данных
с фиксированной частотой.\par Этот тип оконечных точек являются основным транспортным методом не только для
геймпадов, но и для USB клавиатур и мышей.
Передачи данного типа имеют зарезервированную пропускную способность. 

\subsection{Блоки запроса USB}
urb используется для передачи или приёма данных в или из заданной оконечной точки USB
на заданное USB устройство в асинхронном режиме. Каждая оконечная точка в
устройстве может обрабатывать очередь urb-ов, так что перед тем, как очередь опустеет, к
одной оконечной точке может быть отправлено множество urb-ов. Типичный жизненный цикл
urb выглядит следующим образом:
\begin{itemize}
	\item создание драйвером USB;
	\item назначение в определённую оконечную точку заданного USB устройства;
	\item передача драйвером USB устройства в USB ядро;
	\item передача USB ядром в заданный драйвер контроллера USB узла для указанного
	устройства;
	\item обработка драйвером контроллера USB узла, который выполняет передачю по USB в
	устройство;
	\item после завершения работы с urb драйвер контроллера USB узла уведомляет драйвер USB
	устройства.
\end{itemize}


\pagebreak

